\newglossaryentry{awk} {
  name=awk,
  description={ist eine musterorientierte Lese- und Verarbeitungssprache. Sie wird hier benutzt, um die Ausgabe eines \gls{find}-Befehls weiter zu verarbeiten. \cite{manpage_awk}}
}
\newglossaryentry{basecamp} {
  name=Basecamp,
  description={ist ein web-basiertes Projekt Management Tool der Firma 37signals. Damit können To-dos, Dateien, Nachrichten, Zeiten und Meilensteine verwaltet werden. (\url{http://basecamphq.com})}
}
\newglossaryentry{deployment} {
  name=Deployment,
  description={beschreibt alle Aktivitäten, eine Software in den Zustand zu überführen, in dem sie vom Endnutzer benutzt werden kann. Dazu gehören unter Anderem: Release, Installation, Aktivierung, Deaktivierung und Update. \cite{deployment_tech}}
}
\newglossaryentry{feature} {
  name=Feature,
  description={}
}
\newglossaryentry{feature_freeze} {
  name=Feature Freeze,
  description={(auch: Code Freeze) ist ein willkürlich definierter Zeitpunkt im Entwicklungszyklus einer Software, ab dem keine neuen Features mehr entwickelt werden. Bis zum nächsten Release wird ausschließlich an Stabilität gearbeitet. \cite[S.298]{praxiswissen_softwareing}}
}
\newglossaryentry{find} {
  name=find,
  description={ist ein UNIX-Werkzeug, um Dateien in einer Dateihierarchie zu finden. \cite{manpage_find}}
}
\newglossaryentry{git} {
  name=git,
  description={ist ein \gls{dvcs}. Es wurde von Linus Torvalds für die Linux Kernel Entwicklung geschrieben.}
}
\newglossaryentry{github} {
  name=GitHub,
  description={ist eine Hosting Plattform für \gls{git} Repositories. Es gibt sowohl öffentliche als auch private Repositories. Letztere sind für Firmen interessant, da diese ihre Projekte nicht für alle zugänglich haben wollen. Neben dem Hosting bietet die Plattform weitere Funktionalität wie Issue Tracking, News Feeds und grafische Auswertungen von Repositories. (\url{https://github.com/})}.
}
\newglossaryentry{inps} {
  name=inpsyde GmbH,
  description={ist eine deutsch Agentur, die sich auf das \gls{cms} \gls{wordpress} spezialisiert hat. (\url{http://inpsyde.com/})}
}
\newglossaryentry{jenkins} {
  name=Jenkins,
  description={(vorher: Hudson) ist eine in Java geschriebene Open Source \gls{ci} Software. (\url{http://jenkins-ci.org/})}
}
\newglossaryentry{json} {
  name=JSON,
  description={JavaScript Object Notation ist ein textbasiertes, sprachunabhängiges Datenaustauschformat. \cite{json}}
}
\newglossaryentry{mysql} {
  name=MySQL,
  description={ist ein von Oracle entwickeltes Datenbank Management System.}
}
\newglossaryentry{php} {
  name=PHP,
  description={ist eine populäre Scriptsprache zum Erstellen dynamischer Webseiten.}
}
\newglossaryentry{produktivsystem} {
  name=Produktivsystem,
  description={}
}
\newglossaryentry{produkt} {
  name=produkt,
  description={}
}
\newglossaryentry{pull request} {
  name=Pull Request,
  description={}
}
\newglossaryentry{reposerver} {
  name=Repository-Server,
  description={}
}
\newglossaryentry{repository} {
  name=Repository,
  description={}
}
\newglossaryentry{rest} {
  name=REST,
  description={}
}
\newglossaryentry{rsync} {
  name=rsync,
  description={}
}
\newglossaryentry{ror} {
  name=Ruby on Rails,
  description={}
}
\newglossaryentry{ruby} {
  name=Ruby,
  description={}
}
\newglossaryentry{stagingsystem} {
  name=Stagingsystem,
  description={}
}
\newglossaryentry{webhook} {
  name=Webhook,
  description={}
}
\newglossaryentry{wordpress} {
  name=WordPress,
  description={}
}
\newglossaryentry{working directory} {
  name=Working Directory,
  description={}
}
\newglossaryentry{zielsystem} {
  name=Zielsystem,
  description={}
}

\newacronym{ftp}{FTP}{File Transfer Protocol}
\newacronym{cli}{CLI}{Command Line Interface}
\newacronym{sftp}{SFTP}{SSH File Transfer Protocol}
\newacronym{ssh}{SSH}{Secure Shell}
\newacronym{qa}{QA}{Quality Assurance}
\newacronym{ape}{APE}{Automatic Push Environment}
\newacronym{cms}{CMS}{Content Management System}
\newacronym{ci}{CI}{Continuous Integration}
\newacronym{ide}{IDE}{Integrated Development Environment}
\newacronym{dvcs}{DVCS}{Distributed Version Control System}

\makeglossaries