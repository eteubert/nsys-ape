\section{Einleitung} % (fold)
\label{sec:einleitung}

Das Projekt entsteht im Auftrag der Agentur inpsyde GmbH. Diese erhofft sich Kosteinersparnis durch Automatisierung, sowie eine solidere rechtliche Basis gegenüber ihren Kunden.

Automatisierung wird an verschiedenen Stellen eingeführt. Da bereits das Versionskontrollsystem git verwendet wird, werden dessein Eigenschaften ausgenutzt. Hält ein Entwickler ein Feature für abgeschlossen, stellt er über den Service GitHub einen so genannten Pull Request. Alle am Projekt beteiligten Entwickler werden benachrichtigt und können Anmerkungen und sogar Änderungen vornehmen, bevor das Feature gemerged wird. Dies geschieht automatisch. Außerdem wird im Projektmanagement-Tool Basecamp eine zum Pull Request passende ToDo erstellt. Der Kunde wird außerdem per E-Mail benachrichtigt.

Letztgenannter Part ist insbesondere rechtlich interessant. Der Kunde kann das Feature nun in einer separaten Testumgebung ansehen und eventuell von seiner eigenen Q\&A-Abteilung verifizieren lassen. Ist der Kunde zufrieden, hakt er die ToDo ab und gibt damit sein Einverständnis. Es kann vertraglich geregelt werden, dass der Kunde sich der Qualität des Features versichert hat und somit die inpsyde GmbH keinerlei Kosten für eventuell auf diesem Feature basierenden Ausfällen übernimmt.

Die Bestätigung des Kunden löst einen weiteren Automatisierungsmechanismus auf. Das Feature wird in den produktiven Versionierungszweig gemerged und sofort auf das Produktivsystem deployed. Der Entwickler muss nichts weiter unternehmen und der Kunde kann sein akzeptiertes Feature ohne Verzögerung betrachten.

Insgesamt wird ein Service entwickelt, welcher die bereits genutzten Systeme GitHub und Basecamp verbindet und einige Automatisierungen durchführt. Die Integration ist dabei transparent. Weder Kunde noch Entwickler sieht jemals eine grafische Oberfläche des Services. Beide Parteien profitieren jedoch von den Automatisierungen und Absicherungen, welche der Service bietet.

% section einleitung (end)

